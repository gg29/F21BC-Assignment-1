\providecommand{\algname}{exp12{}}
\providecommand{\algfolder}{exp12/}
\providecommand{\bbobpptablecaption}[1]{
%
    Expected running time (ERT in number of function 
    evaluations) divided by the best ERT measured during BBOB-2009. The ERT 
    and in braces, as dispersion measure, the half difference between 90 and 
    10\%-tile of bootstrapped run lengths appear in the second row of each cell,  
    the best ERT
    %
    (preceded by the target \Df-value in \textit{italics}) in the first. 
    \#succ is the number of trials that reached the target value of the last column.
    %
    The median number of conducted function evaluations is additionally given in 
    \textit{italics}, if the target in the last column was never reached. 
    \textbf{Bold} entries are statistically significantly better (according to
    the rank-sum test) compared to the best algorithm in BBOB-2009, with
    $p = 0.05$ or $p = 10^{-k}$ when the number $k > 1$ is following the
    $\downarrow$ symbol, with Bonferroni correction by the number of
    functions.
    
}
\providecommand{\bbobppfigdimlegend}[1]{
%
    Expected number of $f$-evaluations (\ERT, lines) to reach $\fopt+\Df$;
    median number of $f$-evaluations (+) to reach the most difficult
    target that was reached not always but at least once; maximum number of
    $f$-evaluations in any trial ({\color{red}$\times$}); interquartile 
    range with median (notched boxes) of simulated runlengths
    to reach $\fopt+\Df$; all values are divided by dimension and 
    plotted as $\log_{10}$ values versus dimension. %
    %
    Shown is the \ERT\ for 
    targets just not reached by
%    the largest $\Df$-values $\ge10^{-8}$ for which the \ERT\ of 
    the artificial GECCO-BBOB-2009 best algorithm  
    within the given budget $k\times\DIM$, where $k$ is shown in the legend.
%    was above $\{0.5, 1.2, 3, 10, 50\}\times\DIM$ evaluations. 
    Numbers above \ERT-symbols indicate the number of trials reaching the respective target.  
    The light thick line with diamonds indicates the respective best result from BBOB-2009 for 
    the most difficult target. 
    Slanted grid lines indicate a scaling with ${\cal O}(\DIM)$ compared to ${\cal O}(1)$  
    when using the respective 2009 best algorithm. 
    
}
\providecommand{\bbobpprldistrlegend}[1]{
%
     Empirical cumulative distribution functions (ECDF), plotting the fraction of
     trials with an outcome not larger than the respective value on the $x$-axis.
     #1%
     Left subplots: ECDF of number of function evaluations (FEvals) divided by search space dimension $D$,
     to fall below $\fopt+\Df$ where \Df\ is the
     target just not reached by the GECCO-BBOB-2009 best algorithm within a budget of
     % largest $\Df$-value $\ge10^{-8}$ for which the best \ERT\ seen in the GECCO-BBOB-2009 was yet above
     $k\times\DIM$ evaluations, where $k$ is the first value in the legend. %
     Legends indicate for each target the number of functions that were solved in at
     least one trial within the displayed budget.%
     Right subplots: ECDF of the
     best achieved $\Df$
     for running times of $0.5D, 1.2D, 3D, 10D, 100D, 1000D,\dots$
     function evaluations
     (from right to left cycling cyan-magenta-black\dots) and final $\Df$-value (red),
     where \Df\ and \textsf{Df} denote the difference to the optimal function value.
     Light brown lines in the background show ECDFs for the most difficult target of all
     algorithms benchmarked during BBOB-2009.
}
\providecommand{\bbobloglossfigurecaption}[1]{
%
    \ERT\ loss ratios (see Figure~\ref{tab:ERTloss} for details).  
    Each cross ({\color{blue}$+$}) represents a single function, the line
    is the geometric mean.
    
}
\providecommand{\bbobloglosstablecaption}[1]{
% \bbobloglosstablecaption{}
    \ERT\ loss ratio versus the budget in number of $f$-evaluations
    divided by dimension.
    For each given budget \FEvals, the target value \ftarget\ is computed
    as the best target $f$-value reached within the
    budget by the given algorithm.
    Shown is then the \ERT\ to reach \ftarget\ for the given algorithm
    or the budget, if the GECCO-BBOB-2009 best algorithm
    reached a better target within the budget,
    divided by the best \ERT\
    seen in GECCO-BBOB-2009 to reach \ftarget.
    Line: geometric mean. Box-Whisker error bar: 25-75\%-ile with median
    (box), 10-90\%-ile (caps), and minimum and maximum \ERT\ loss ratio
    (points). The vertical line gives the maximal number of function evaluations
    in a single trial in this function subset. See also
    Figure~\ref{fig:ERTlogloss} for results on each function subgroup.
    
}
\providecommand{\algname}{exp11{}}
\providecommand{\algfolder}{exp11/}
\providecommand{\bbobpptablecaption}[1]{
%
    Expected running time (ERT in number of function 
    evaluations) divided by the best ERT measured during BBOB-2009. The ERT 
    and in braces, as dispersion measure, the half difference between 90 and 
    10\%-tile of bootstrapped run lengths appear in the second row of each cell,  
    the best ERT
    %
    (preceded by the target \Df-value in \textit{italics}) in the first. 
    \#succ is the number of trials that reached the target value of the last column.
    %
    The median number of conducted function evaluations is additionally given in 
    \textit{italics}, if the target in the last column was never reached. 
    \textbf{Bold} entries are statistically significantly better (according to
    the rank-sum test) compared to the best algorithm in BBOB-2009, with
    $p = 0.05$ or $p = 10^{-k}$ when the number $k > 1$ is following the
    $\downarrow$ symbol, with Bonferroni correction by the number of
    functions.
    
}
\providecommand{\bbobppfigdimlegend}[1]{
%
    Expected number of $f$-evaluations (\ERT, lines) to reach $\fopt+\Df$;
    median number of $f$-evaluations (+) to reach the most difficult
    target that was reached not always but at least once; maximum number of
    $f$-evaluations in any trial ({\color{red}$\times$}); interquartile 
    range with median (notched boxes) of simulated runlengths
    to reach $\fopt+\Df$; all values are divided by dimension and 
    plotted as $\log_{10}$ values versus dimension. %
    %
    Shown is the \ERT\ for 
    targets just not reached by
%    the largest $\Df$-values $\ge10^{-8}$ for which the \ERT\ of 
    the artificial GECCO-BBOB-2009 best algorithm  
    within the given budget $k\times\DIM$, where $k$ is shown in the legend.
%    was above $\{0.5, 1.2, 3, 10, 50\}\times\DIM$ evaluations. 
    Numbers above \ERT-symbols indicate the number of trials reaching the respective target.  
    The light thick line with diamonds indicates the respective best result from BBOB-2009 for 
    the most difficult target. 
    Slanted grid lines indicate a scaling with ${\cal O}(\DIM)$ compared to ${\cal O}(1)$  
    when using the respective 2009 best algorithm. 
    
}
\providecommand{\bbobpprldistrlegend}[1]{
%
     Empirical cumulative distribution functions (ECDF), plotting the fraction of
     trials with an outcome not larger than the respective value on the $x$-axis.
     #1%
     Left subplots: ECDF of number of function evaluations (FEvals) divided by search space dimension $D$,
     to fall below $\fopt+\Df$ where \Df\ is the
     target just not reached by the GECCO-BBOB-2009 best algorithm within a budget of
     % largest $\Df$-value $\ge10^{-8}$ for which the best \ERT\ seen in the GECCO-BBOB-2009 was yet above
     $k\times\DIM$ evaluations, where $k$ is the first value in the legend. %
     Legends indicate for each target the number of functions that were solved in at
     least one trial within the displayed budget.%
     Right subplots: ECDF of the
     best achieved $\Df$
     for running times of $0.5D, 1.2D, 3D, 10D, 100D, 1000D,\dots$
     function evaluations
     (from right to left cycling cyan-magenta-black\dots) and final $\Df$-value (red),
     where \Df\ and \textsf{Df} denote the difference to the optimal function value.
     Light brown lines in the background show ECDFs for the most difficult target of all
     algorithms benchmarked during BBOB-2009.
}
\providecommand{\bbobloglossfigurecaption}[1]{
%
    \ERT\ loss ratios (see Figure~\ref{tab:ERTloss} for details).  
    Each cross ({\color{blue}$+$}) represents a single function, the line
    is the geometric mean.
    
}
\providecommand{\bbobloglosstablecaption}[1]{
% \bbobloglosstablecaption{}
    \ERT\ loss ratio versus the budget in number of $f$-evaluations
    divided by dimension.
    For each given budget \FEvals, the target value \ftarget\ is computed
    as the best target $f$-value reached within the
    budget by the given algorithm.
    Shown is then the \ERT\ to reach \ftarget\ for the given algorithm
    or the budget, if the GECCO-BBOB-2009 best algorithm
    reached a better target within the budget,
    divided by the best \ERT\
    seen in GECCO-BBOB-2009 to reach \ftarget.
    Line: geometric mean. Box-Whisker error bar: 25-75\%-ile with median
    (box), 10-90\%-ile (caps), and minimum and maximum \ERT\ loss ratio
    (points). The vertical line gives the maximal number of function evaluations
    in a single trial in this function subset. See also
    Figure~\ref{fig:ERTlogloss} for results on each function subgroup.
    
}
\providecommand{\algname}{exp10{}}
\providecommand{\algfolder}{exp10/}
\providecommand{\bbobpptablecaption}[1]{
%
    Expected running time (ERT in number of function 
    evaluations) divided by the best ERT measured during BBOB-2009. The ERT 
    and in braces, as dispersion measure, the half difference between 90 and 
    10\%-tile of bootstrapped run lengths appear in the second row of each cell,  
    the best ERT
    %
    (preceded by the target \Df-value in \textit{italics}) in the first. 
    \#succ is the number of trials that reached the target value of the last column.
    %
    The median number of conducted function evaluations is additionally given in 
    \textit{italics}, if the target in the last column was never reached. 
    \textbf{Bold} entries are statistically significantly better (according to
    the rank-sum test) compared to the best algorithm in BBOB-2009, with
    $p = 0.05$ or $p = 10^{-k}$ when the number $k > 1$ is following the
    $\downarrow$ symbol, with Bonferroni correction by the number of
    functions.
    
}
\providecommand{\bbobppfigdimlegend}[1]{
%
    Expected number of $f$-evaluations (\ERT, lines) to reach $\fopt+\Df$;
    median number of $f$-evaluations (+) to reach the most difficult
    target that was reached not always but at least once; maximum number of
    $f$-evaluations in any trial ({\color{red}$\times$}); interquartile 
    range with median (notched boxes) of simulated runlengths
    to reach $\fopt+\Df$; all values are divided by dimension and 
    plotted as $\log_{10}$ values versus dimension. %
    %
    Shown is the \ERT\ for 
    targets just not reached by
%    the largest $\Df$-values $\ge10^{-8}$ for which the \ERT\ of 
    the artificial GECCO-BBOB-2009 best algorithm  
    within the given budget $k\times\DIM$, where $k$ is shown in the legend.
%    was above $\{0.5, 1.2, 3, 10, 50\}\times\DIM$ evaluations. 
    Numbers above \ERT-symbols indicate the number of trials reaching the respective target.  
    The light thick line with diamonds indicates the respective best result from BBOB-2009 for 
    the most difficult target. 
    Slanted grid lines indicate a scaling with ${\cal O}(\DIM)$ compared to ${\cal O}(1)$  
    when using the respective 2009 best algorithm. 
    
}
\providecommand{\bbobpprldistrlegend}[1]{
%
     Empirical cumulative distribution functions (ECDF), plotting the fraction of
     trials with an outcome not larger than the respective value on the $x$-axis.
     #1%
     Left subplots: ECDF of number of function evaluations (FEvals) divided by search space dimension $D$,
     to fall below $\fopt+\Df$ where \Df\ is the
     target just not reached by the GECCO-BBOB-2009 best algorithm within a budget of
     % largest $\Df$-value $\ge10^{-8}$ for which the best \ERT\ seen in the GECCO-BBOB-2009 was yet above
     $k\times\DIM$ evaluations, where $k$ is the first value in the legend. %
     Legends indicate for each target the number of functions that were solved in at
     least one trial within the displayed budget.%
     Right subplots: ECDF of the
     best achieved $\Df$
     for running times of $0.5D, 1.2D, 3D, 10D, 100D, 1000D,\dots$
     function evaluations
     (from right to left cycling cyan-magenta-black\dots) and final $\Df$-value (red),
     where \Df\ and \textsf{Df} denote the difference to the optimal function value.
     Light brown lines in the background show ECDFs for the most difficult target of all
     algorithms benchmarked during BBOB-2009.
}
\providecommand{\bbobloglossfigurecaption}[1]{
%
    \ERT\ loss ratios (see Figure~\ref{tab:ERTloss} for details).  
    Each cross ({\color{blue}$+$}) represents a single function, the line
    is the geometric mean.
    
}
\providecommand{\bbobloglosstablecaption}[1]{
% \bbobloglosstablecaption{}
    \ERT\ loss ratio versus the budget in number of $f$-evaluations
    divided by dimension.
    For each given budget \FEvals, the target value \ftarget\ is computed
    as the best target $f$-value reached within the
    budget by the given algorithm.
    Shown is then the \ERT\ to reach \ftarget\ for the given algorithm
    or the budget, if the GECCO-BBOB-2009 best algorithm
    reached a better target within the budget,
    divided by the best \ERT\
    seen in GECCO-BBOB-2009 to reach \ftarget.
    Line: geometric mean. Box-Whisker error bar: 25-75\%-ile with median
    (box), 10-90\%-ile (caps), and minimum and maximum \ERT\ loss ratio
    (points). The vertical line gives the maximal number of function evaluations
    in a single trial in this function subset. See also
    Figure~\ref{fig:ERTlogloss} for results on each function subgroup.
    
}
\providecommand{\algname}{exp9{}}
\providecommand{\algfolder}{exp9/}
\providecommand{\bbobpptablecaption}[1]{
%
    Expected running time (ERT in number of function 
    evaluations) divided by the best ERT measured during BBOB-2009. The ERT 
    and in braces, as dispersion measure, the half difference between 90 and 
    10\%-tile of bootstrapped run lengths appear in the second row of each cell,  
    the best ERT
    %
    (preceded by the target \Df-value in \textit{italics}) in the first. 
    \#succ is the number of trials that reached the target value of the last column.
    %
    The median number of conducted function evaluations is additionally given in 
    \textit{italics}, if the target in the last column was never reached. 
    \textbf{Bold} entries are statistically significantly better (according to
    the rank-sum test) compared to the best algorithm in BBOB-2009, with
    $p = 0.05$ or $p = 10^{-k}$ when the number $k > 1$ is following the
    $\downarrow$ symbol, with Bonferroni correction by the number of
    functions.
    
}
